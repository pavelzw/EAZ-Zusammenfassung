\section{Teilbarkeitslehre, Primelemente}

\subsection{Teilbarkeit}

\begin{karte}{Teilbarkeit}
    Sei \(R\) ein kommutativer Ring. Dann heißt \(a\in R\) ein 
    \textit{Teiler} von \(b\in R\), falls ein \(c\in R\) 
    existiert, sodass \(b=c\cdot a\). Wir schreiben dann wieder 
    \( a \;|\; b \) oder manchmal auch \( a \;|_R \; b \).
\end{karte}

\begin{karte}{Assoziiertheit}
    Sei \(R\) ein kommutativer Ring. Zwei Elemente \(a,b\in R\) 
    heißen \textit{assoziiert}, falls eine Einheit \(e\in R^\times\) 
    existiert, sodass \(b = a \cdot e\). 
    Für \(R=\Z\) bedeutet das Gleichheit bis aufs Vorzeichen.\\
    Wenn \(R\) kommutativ und nullteilerfrei ist, dann wird durch 
    die Teilbarkeit eine Ordnungsrelation auf der Menge der Assoziiertenklassen 
    festgelegt: \[ a R^\times \preceq b R^\times \Leftrightarrow a \;|\; b. \]
\end{karte}

\begin{karte}{\( \ggT \), teilerfremd}
    Sei \(R\) ein kommutativer und  nullteilerfreier Ring und \(a,b\in R\).
    \begin{enumerate}
        \item Das Element \(g\in R\) heißt \textit{größter gemeinsamer Teiler} 
        von \(a\) und \(b\), wenn \(g\) ein gemeinsamer Teiler ist und jeder 
        gemeinsame Teiler von \(a\) und \(b\) auch \(g\) teilt.
        \item \(a\) und \(b\) heißen \textit{teilerfremd}, wenn die einzigen 
        gemeinsamen Teiler die Einheiten in \(R\) sind. 
    \end{enumerate}
\end{karte}

\begin{karte}{Idealisierung}
    Sei \(R\) ein nullteilerfreier kommutativer Ring. Weiter seien \(a,b\in R\). 
    \begin{enumerate}
        \item Ist \(d\) ein gemeinsamer Teiler von \(a\) und \(b\), so teilt \(d\) 
        auch jede Linearkombination \(ax + by, x,y\in R\).
        \item Wenn es ein \(g\in R\) gibt, sodass 
        \[ \set{ ax+by \;|\; x,y\in R } = Rg := \set{rg \;|\; r\in R} \]
        gilt, dann ist \(g\) ein \(\ggT\) von \(a\) und \(b\).
    \end{enumerate}
\end{karte}

\begin{karte}{Hauptideal}
    Ein Ideal \(I \subseteq R\) heißt \textit{Hauptideal}, falls ein 
    \(g\in I\) existiert, sodass \(I = Rg\) gilt. Falls der Ring klar ist, 
    werden wir oft \((g) := Rg\) schreiben. \\
    Ein nullteilerfreier kommutativer Ring \(R\), in dem jedes Ideal 
    ein Hauptideal ist, heißt \textit{Hauptidealring}.
\end{karte}

\begin{karte}{Assoziiertenklassen und Ideale}
    Sei \(R\) ein Hauptidealring. Dann gelten: 
    \begin{enumerate}
        \item Zwei Elemente \(g,h\in R\) sind genau dann Erzeuger desselben 
        Hauptideals \(Rg = Rh\), wenn sie assoziiert sind. 
        \item In jeder nichtleeren Teilmenge \(S\subseteq R\) gibt es ein 
        Element \(m\), das bezüglich Teilbarkeit minimal ist.
    \end{enumerate}
\end{karte}

\begin{karte}{Euklidischer Ring}
    Sei \(R\) ein nullteilerfreier kommutativer Ring. Weiter sei 
    \(\abb{\gamma}{R}{\N_0}\) eine Abbildung. \\
    Dann heißt \( (R,\gamma) \) ein \textit{euklidischer Ring}, 
    falls \( \gamma(r) = 0 \Leftrightarrow r = 0 \) 
    und vor allem folgendes gilt: Für \(a,b\in R, b\neq 0\) gibt 
    es ein \(c \in R\), sodass 
    \[ \gamma(a - bc) < \gamma(b). \]
    Man hat hier also eine quantitative Version einer Division 
    mit Rest, und dies führt zu ähnlichen Möglichkeiten wie bei 
    den ganzen Zahlen.\\
    Jeder euklidische Ring ist ein Hauptidealring.
\end{karte}

\subsection{Arithmetik in Hauptidealringen}

\begin{karte}{Irreduzibel, prim}
    Sei \(R\) ein kommutativer Ring. \\
    Ein Element \(m\in R\) heißt \textit{irreduzibel}, 
    wenn \(m\notin R^\times \) und für alle \(a,b\in R\) gilt: 
    \[ m = ab \Rightarrow a\in R^\times \vee b\in R^\times. \]
    Ein Element \(p\in R\) heißt \textit{Primelement}, wenn \(p\notin R^\times\)
    und wenn für alle \(a,b\in R\) gilt: 
    \begin{center}
        \(p\) teilt \(ab\) \(\Rightarrow\) \(p\) teilt \(a\) oder \(b\).
    \end{center}
    Irreduzibilität eines Elementes \(m\in R\) heißt also, dass 
    seine Assoziiertenklasse \(m R^\times\) in \(R\) bezüglich der 
    Ordnungsrelation der Teilbarkeit minimal ist: Jeder Teiler 
    von \(m\) ist entweder eine Einheit oder zu \(m\) assoziiert. \\
    Das Nullelement eines Rings \(R\) ist niemals irreduzibel. Es ist 
    prim genau dann, wenn \(R\) nullteilerfrei ist. 
\end{karte}

\begin{karte}{Prim vs. irreduzibel}
    Sei \(R\) ein nullteilerfreier kommutativer Ring. 
    \begin{enumerate}
        \item Ein von \(0\) verschiedenes Primelement in \(R\) ist immer irreduzibel. 
        \item Wenn \(R\) ein Hauptidealring ist, dann ist ein irreduzibles Element in \(R\) immer auch prim.
    \end{enumerate}
\end{karte}

\begin{karte}{Primzerlegung in Hauptidealringen}
    Sei \(R\) eine Hauptidealring. Weiter sei \(\P_R\) ein 
    Vertretersystem der Assoziiertenklassen von Primelementen \(\neq 0\). \\
    Dann ist jedes \(r\in R \setminus \set{0}\) assoziiert zu einem Produkt 
    von endlich vielen Elementen in \(\P_R\).\\
    Sind weiter \(s,t\in \N_0\) und \(p_1, \ldots, p_s, q_1, \ldots, q_t \in \P_R\) 
    derart, dass Einheiten \(\delta, \varepsilon \in R^\times\) 
    existieren mit 
    \[ r = \delta \cdot p_1 \cdots p_s = \varepsilon \cdot q_1 \cdots q_t, \]
    so gelten \( \varepsilon = \delta, s = t \) und -- bis auf eine 
    Vertauschung der Reihenfolge der Faktoren -- es gilt \(p_i = q_i\) 
    für alle \(1 \leq i \leq s\).
\end{karte}

\begin{karte}{Summen zweier Quadrate}
    Eine natürliche Zahl \(n\) ist genau dann als Summe zweier Quadrate 
    von ganzen Zahlen schreibbar, wenn ihr quadratfreier Anteil keinen 
    Primteiler hat, der bei Division durch \(4\) Rest \(3\) lässt.
\end{karte}

\begin{karte}{Primideal}
    Sei \(R\) ein kommutativer Ring. Ein Ideal \(I \subset R\) 
    heißt \textit{maximales Ideal}, wenn \(I \neq R\) und wenn 
    zwischen \(I\) und \(R\) kein weiteres Ideal dazwischenliegt. \\
    Äquivalent dazu ist, dass \(R/I\) ein Körper ist.\\
    Ein Ideal \( I \subset R \) heißt \textit{Primideal}, falls für 
    alle \(x,y\in R\) gilt: 
    \[ xy \in I \Rightarrow x\in I \vee y \in I. \]
    Bei Hauptidealringen, die keine Körper sind, fallen die von \(\set{0}\) 
    verschiedenen Primideale mit den maximalen Idealen zusammen, da beide von 
    Null verschiedenen Primelementen = irreduziblen Elementen erzeugt werden.\\
    \(I\) ist genau dann ein Primideal, wenn \(R/I\) ein Integritätsbereich ist.
\end{karte}

\subsection{Gleichungssysteme}

\begin{karte}{Basen}
    Es sei \(R\) ein kommutativer Ring und \(M\) ein \(R\)-Modul. 
    Dann heißt \(B\subseteq M\) eine \(R\)-Basis von \(M\), 
    wenn sich jedes \(m\in M\) auf eindeutig bestimmte Art als 
    \[ m = \sum_{b\in B} \lambda_b \cdot b, \lambda_b \in R, \text{ fast alle } \lambda_b = 0 \]
    schreiben lässt.\\
    Wenn \(M\) eine Basis \(B\) hat, dann nennt man \(M\) auch einen \textit{freien \(R\)-Modul}.\\
    Jeder freie \(R\)-Modul \(M\) mit einer endlichen Basis ist zu \(R^r\) 
    isomorph, wir schreiben die Anzahl der Elemente einer Basis \(r\)
    als \textit{Rang} von \(M\).
\end{karte}

\begin{karte}{Basis von Modul}
    Seien \(R\) ein Hauptidealring, \(n\in \N_0\) und 
    \(M\subseteq R^n\) ein Untermodul. \\
    Dann hat \(M\) eine Basis aus höchstens \(n\) Elementen. 
\end{karte}

\begin{karte}{Unimodulare Matrizen}
    Sei \(R\) ein kommutativer Ring und \(M \in R^{n\times n}\) gegeben. 
    Dann ist äquivalent
    \begin{enumerate}
        \item Die Spalten von \(M\) bilden eine \(R\)-Basis von \(R^n\).
        \item Es gibt eine zu \(M\) inverse Matrix mit Einträgen in \(R\).
        \item \(\det(M) \in R^\times\).
    \end{enumerate}
    Matrizen, für die eine dieser Aussagen stimmt, heißen \textit{unimodulare Matrizen}.
\end{karte}

\begin{karte}{Inhalt}
    Sei \(R\) ein Hauptidealring und \(v\in R^n\). Dann heißt der \(\ggT\)
    der Einträge von \(v\) der \textit{Inhalt} von \(v\), kurz \(\Inh(v)\).\\
    Wenn \(v\) Inhalt \(1\) hat, dann heißt \(v\) auch ein \textit{primitiver Vektor}.
\end{karte}

\begin{karte}{Basisergänzungssatz}
    Sei \(R\) ein Hauptidealring. Ein Vektor \(v\in R^n\) ist genau dann 
    ein Element einer Basis von \(R^n\), wenn \(\Inh(v) = 1\).
\end{karte}

\begin{karte}{Elementarteilersatz}
    Seien \(R\) ein Hauptidealring und \(F\) ein freier \(R\)-Modul 
    vom Rang \(n\) sowie \(U \subseteq F\) ein Untermodul vom Rang \(r\).\\
    Dann gibt es eine Basis \(\set{ b_1, \ldots, b_n }\) von \(F\) und 
    Elemente \( e_1 \;|\; e_2 \;|\; \cdots \;|\; e_r \in R \), 
    sodass 
    \[ \set{ e_1 b_1, \ldots, e_r b_r } \] 
    eine Basis von \(U\) ist.\\
    Die Elemente \(e_1, \ldots, e_r\) sind bis auf Assoziiertheit 
    eindeutig durch \(U\) festgelegt. 
\end{karte}

\begin{karte}{Matrixversion Elementarteilersatz}
    Sei \(M\in R^{n\times m}\) eine ganzzahlige Matrix. Dann gibt es unimodulare
    Matrizen \(S \in \GL_n(R), T\in \GL_m(R)\), sodass 
    \[ S^{-1} M T = \begin{pmatrix}
        D & 0 \\ 0 & 0
    \end{pmatrix}, D = \diag(e_1, \ldots, e_r), e_1 \;|\; e_2 \;|\; \cdots \;|\; e_r \neq 0. \]
\end{karte}

\begin{karte}{Lineare Gleichungssysteme}
    Will man \(Mx = b\) lösen, löst man stattdessen 
    \[ S^{-1} M T y = S^{-1} b \]
    und rechnet diesen Lösungsraum zurück mithilfe \(T^{-1}\).
\end{karte}

\begin{karte}{Struktursatz für endlich erzeugte abelsche Gruppen}
    Jede endlich erzeugte abelsche Gruppe \(A\) ist ein direktes Produkt 
    von zyklischen Gruppen. \\
    Genauer gibt es natürliche Zahlen \(e_1 \;|\; e_2 \;|\; \cdots \;|\; e_s\) 
    und \(r\in \N_0\), sodass 
    \[ A \cong \Z/(e_1) \times \cdots \times \Z/(e_s) \times \Z^r. \]
\end{karte}

\begin{karte}{Einheitengruppen von Körpern}
    Sei \(K\) ein Körper und \(G\subset K^\times \) eine endliche 
    Untergruppe seiner Einheitengruppe. Dann ist \(G\) zyklisch.
\end{karte}
\section{Euklid, Teilbarkeit}

\subsection{Teilbarkeit}

\begin{karte}{Teiler, \(\ggT, \kgV\), Teilerfremdheit}
    \(n\) eine natürliche Zahl. \(d\in \N\) ist ein \textit{Teiler} 
    von \(n\), falls \(t\in \N\) existiert mit \(d\cdot t = n\). Schreibe \(d \;|\; n\).
    Dann heißt \(n\) \textit{Vielfaches} von \(d\).

    Für zwei Zahlen \(m,n\) ist die Menge aller gemeinsamen Teiler endlich 
    und nicht leer. Das größte Element dieser Menge heißt 
    \textit{größter gemeinsamer Teiler}: \(\ggT(m,n)\) oder \((m,n)\).

    In der Menge aller gemeinsamen Vielfachen von \(m\) und \(n\) 
    liegt \(m\cdot n\). Also gibt es ein kleinstes Element dieser Menge, 
    das \textit{kleinste gemeinsame Vielfache} von \(m\) und \(n\) 
    wird mit \(\kgV(m,n)\) notiert. 

    \(m\) und \(n\) sind \textit{teilerfremd}, wenn der einzige gemeinsame Teiler 
    in \(\N\) \(1\) ist.

    Es gilt \(\ggT(0,m) = m\). \(\ggT(|m|,|n|) = \ggT(m,n)\).
\end{karte}

\begin{karte}{\(\ggT\) als Linearkombination}
    Es seien \(m,n\in \Z\) gegeben. Dann gibt es \(c,d\in \Z\) mit 
    \[ mc+nd=\ggT(m,n). \]
\end{karte}

\begin{karte}{Teiler des \(\ggT\)}
    Für ganze Zahlen \(m,n\) sind die Teiler von \(\ggT(m,n)\) genau die gemeinsamen 
    Teiler von \(m\) und \(n\).
\end{karte}

\begin{karte}{Division mit Rest}
    Für jede ganze Zahl \(k\) und jede natürliche Zahl \(n\) gibt es 
    eindeutig bestimmte Zahlen \(f\in \Z\) und \(r\in \set{0,1,\ldots, n-1}\) mit 
    \[ k=fn+r. \]
\end{karte}

\begin{karte}{Kongruenz modulo \(n\)}
    Sei \(n\in\Z\) gegeben. Zwei Zahlen \(k,l\in \Z\) heißen 
    \textit{kongruent modulo} \(n\), wenn \(n\) ein Teiler von \(k-l\) ist. 
    Wir schreiben dann 
    \[ k\equiv l\ (\mathrm{mod}\ n). \]
\end{karte}

\begin{karte}{\(\ggT\) Folgerungen}
    Seien \(m,n\in\N\) gegeben.
    \begin{enumerate}
        \item Für \(g:= \ggT(m,n)\) sind die natürlichen Zahlen \(m/g\) und \(n/g\) teilerfremd.
        \item Wenn \(m,n\) teilerfremd sind und \(u\in \N\) eine Zahl ist, sodass \(m\;|\; nu\) gilt, 
        dann teilt \(m\) schon \(u\).
        \item Es gilt \(\kgV(m,n) \cdot \ggT(m,n) = m\cdot n\).
    \end{enumerate}
\end{karte}

\begin{karte}{Gekürzte Brüche}
    Jede rationale Zahl \(q\) lässt sich auf genau eine Art als 
    \[ q=\frac{z}{n}, z\in \Z, n\in \N, \]
    schreiben, wobei \(z\) und \(n\) teilerfremd sind.
\end{karte}

\subsection{Primzahlen}

\begin{karte}{Primzahl}
    Eine \textit{Primzahl} ist eine natürliche Zahl \(p>1\), die 
    sich nicht als Produkt zweier natürlicher Zahlen schreiben lässt, 
    also keinen natürlichen Teiler außer \(1\) und \(p\) hat. 
    Die Menge der Primzahlen notieren wir mit \(\P\).
\end{karte}

\begin{karte}{Alternative Charakterisierung}
    Es sei \(n>1\) eine natürliche Zahl. Dann sind äquivalent: 
    \begin{enumerate}
        \item \(n\) ist eine Primzahl.
        \item Für jedes Paar \((a,b)\in \N^2\) von natürlichen Zahlen gilt:
        \begin{center}
            \(n\) teilt \(ab\) \(\Rightarrow\) \(n\) teilt \(a\) oder \(n\) teilt \(b\).
        \end{center}
    \end{enumerate}
\end{karte}

\begin{karte}{Fundamentalsatz der Arithmetik}
    Jede natürliche Zahl \(n\) lässt sich als Produkt von Primzahlen schreiben. 
    Diese Darstellung ist eindeutig, wenn die Primfaktoren der Größe nach sortiert werden.
\end{karte}

\begin{karte}{\(p\)-adische Bewertung}
    Es sei \(p\in \P\) eine Primzahl. Dann gibt es für jede ganze Zahl \(k\neq 0\)
    eine eindeutig bestimmte Zahl \(\vp(k)\in \N_0\), sodass \(p^{\vp(k)}\) 
    ein Teiler von \(k\) ist, aber \(p^{\vp(k)+1}\) nicht. 

    Dann gilt insbesondere 
    \[ k=\pm \prod_{p\in\P} p^{\vp(k)}. \]
    Für \(k=0\) schreibt man formal \(\vp(0)=\infty\).

    Es gelten \(\forall k,l\in\Z\) die Regeln 
    \begin{align*}
        \vp(k+l) &\geq \min \set{\vp(k), \vp(l)}, \\
        \vp(k\cdot l) &= \vp(k) + \vp(l).
    \end{align*}
    \(\vp(k)\) heißt die \(p\)-adische Bewertung von \(k\).
\end{karte}

\begin{karte}{\(\vp\) und der \(\ggT\)}
    Es seien \(a,b\in \N\). Dann gelten:
    \begin{enumerate}
        \item \(b\) teilt \(a\) genau dann, wenn 
        \[ \forall p\in\P: \vp(b)\leq \vp(a). \]
        \item Der \(\ggT\) von \(a\) und \(b\) ist 
        \[ g=\prod_{p\in \P} p^{e_p}, e_p = \min \set{\vp(a),\vp(b)}. \]
        \item Das \(\kgV\) von \(a\) und \(b\) ist 
        \[ k=\prod_{p\in \P} p^{f_p}, f_p = \max \set{\vp(a),\vp(b)}. \]
    \end{enumerate}
\end{karte}

\begin{karte}{\(p\)-adische Bewertung auf \(\Q\)}
    Die Abbildung \(\abb{\vp}{\Z}{\N_0\cup \set{\infty}}\) heißt die \textit{\(p\)-adische Bewertung} auf 
    \(Z\). Sie wird durch 
    \[ \vp(\frac{z}{n}) := \vp(z) - \vp(n) \] 
    zu einer Abbildung von \(\Q\) nach \(\Z\cup \set{\infty}\) fortgesetzt und behält 
    ihre Eigenschaften bei. 
\end{karte}

\begin{karte}{Kleiner Satz von Fermat}
    Es sei \(p\) eine Primzahl und \(c\in \Z\). Dann ist \(p\) ein Teiler von 
    \(c^p - c\). 
    
    Alternativ: Falls \(c\) kein Vielfaches von \(p\) ist, gilt 
    \[ c^{p-1} \equiv 1\ (\mathrm{mod}\ p). \]
\end{karte}

\subsection{Verteilung Primzahlen}

\begin{karte}{Anzahl Primzahlen}
    Es gibt unendlich viele Primzahlen.
\end{karte}

\begin{karte}{Lückenhaft}
    Sei \(k\in\N\). Dann gibt es eine natürliche Zahl \(M\), sodass zwischen 
    \(M\) und \(M+k\) keine Primzahl liegt. (\(M=(k+2)! + 2\))
\end{karte}

\begin{karte}{Harmonische Reihe Primzahlen}
    Für jede reelle Zahl \(x > 1\) gilt 
    \[ \sum_{p\in \P, p\leq x} \frac{1}{p} \geq \log(\log x) - \log 2. \]
\end{karte}

\begin{karte}{Lückenlos}
    Sei \(\varepsilon > 0\) gegeben. Dann gibt es eine reelle Zahl \(x_0\), sodass 
    für alle \(x\geq x_0\) im Intervall \([x,(1+\varepsilon)x]\) eine Primzahl existiert.
\end{karte}

\begin{karte}{Dichtheitssatz}
    Die Menge aller Brüche \(p/l\), wobei \(p\) und \(l\) Primzahlen sind, ist 
    dicht in \(\R_{\geq 0}\).
\end{karte}
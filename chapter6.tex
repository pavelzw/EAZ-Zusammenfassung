\section{Körpererweiterungen}

\subsection{Algebraizität}

\begin{karte}{Algebraisch und transzendent}
    Sei \(K\) ein Körper und \(L\) ein Körper, der \(K\) umfasst. 
    Wir nennen \(K \subseteq L\) eine \textit{Körpererweiterung}.
    \begin{enumerate}
        \item Ein Element \(\alpha \in L\) heißt \textit{algebraisch über \(K\)}, 
        falls es ein von Null verschiedenes Polynom \(f \in K[X]\) mit \(f(\alpha) = 0\) gibt.
        \item Ein Element \(\alpha \in L\), das nicht über \(K\) algebraisch 
        ist, heißt \textit{transzendent über \(K\)}.
        \item \(L\) heißt algebraisch über \(K\), wenn jedes Element von \(L\) 
        über \(K\) algebraisch ist. 
        \item Es sei \(\alpha \in L\) über \(K\) algebraisch. Dann ist das 
        \textit{Verschwindungsideal} 
        \[ I(\alpha) := \set{f\in K[X] \;|\; f(\alpha) = 0} \]
        nicht das Nullideal im Polynomring. Der normierte Erzeuger 
        von \(I(\alpha)\) heißt das \textit{Minimalpolynom} von \(\alpha\).
    \end{enumerate}
\end{karte}

\begin{karte}{Adjunktion}
    Den kleinsten Teilkörper von \(L\), der \(K\) und 
    ein gegebenes \(\alpha\) von \(L\) enthält, bezeichnen wir mit 
    \(K(\alpha)\), er entsteht durch \textit{Adjunktion} von \(\alpha\) 
    zu \(K\).\\
    Wenn \(\alpha\) algebraisch ist und \(d\) der Grad des Minimalpolynoms 
    \(m_\alpha\), dann gilt 
    \[ K(\alpha) \cong K[X]/(m_\alpha). \]
    Wenn \(\alpha\) transzendent ist, dann ist 
    \[ K(\alpha) \cong K(X) \]
    isomorph zum Körper der rationalen Funktionen.
\end{karte}

\begin{karte}{Algebraische Erweiterung}
    Sei \(K \subseteq L\) eine Körpererweiterung. Dann gelten 
    \begin{enumerate}
        \item Ein \(\alpha \in L\) ist genau dann über \(K\) algebraisch, 
        wenn die Dimension von \(K(\alpha)\) als \(K\)-Vektorraum endlich ist.
        \item Die Menge aller über \(K\) algebraischen \(\alpha \in L\) ist 
        ein Teilkörper von \(L\).
        \item Sein \(K \subseteq L\) und \(L \subseteq M\) algebraische 
        Körpererweiterungen, so ist auch die Erweiterung \(K \subseteq M\) algebraisch.
    \end{enumerate}
\end{karte}

\begin{karte}{Grad der Körpererweiterung}
    Sei \(K \subseteq L\) eine Körpererweiterung. Die Dimension 
    von \(L\) als \(K\)-Vektorraum nennt man auch den 
    \textit{Grad von \(L\) über \(K\)}: \([L:K]\).\\
    Es ist \(\alpha\) genau dann algebraisch, wenn 
    \([K(\alpha) : K] < \infty\).\\
    Man sagt auch, \(L\) sei eine endliche Erweiterung von \(K\), 
    wenn der Grad endlich ist. Sind \( K \subseteq L \subseteq M \) 
    endliche Körpererweiterungen, so gilt 
    \[ [M:K] = [M:L] \cdot [L:K]. \]
    Wenn \(B\) eine \(K\)-Basis von \(L\) ist und \(C\) 
    eine \(L\)-Basis von \(M\), dann ist \( \set{bc \;|\; b\in B, c\in C} \) 
    eine \(K\)-Basis von \(M\).\\
    Der Grad von \(K(\alpha)\) über \(K\) ist gleich dem Grad des Minimalpolynoms 
    von \(\alpha\) über \(K\). 
\end{karte}

\begin{karte}{Fundamentalkonstruktion}
    Seien \(K\) ein Körper und \(f\in K[X]\) ein normiertes Polynom. \\
    Dann gibt es einen Erweiterungskörper \(L\) von \(K\), über dem \(f\) 
    in Linearfaktoren zerfällt.
\end{karte}

\begin{karte}{Algebraischer Abschluss}
    Ein Körper \(K\) heißt \textit{algebraisch abgeschlossen}, 
    wenn er keine echten algebraischen Erweiterungskörper besitzt. 
    Das ist äquivalent dazu, dass jedes nichtkonstante Polynom 
    in \(K[X]\) mindestens eine Nullstelle in \(K\) hat. \\
    Ein algebraischer Erweiterungskörper von \(K\), der algebraisch 
    abgeschlossen ist, heißt ein \textit{algebraischer Abschluss von \(K\)}. 
    Dieser ist bis auf \(K\)-Algebrenisomorphismen eindeutig bestimmt.
\end{karte}

\subsection{Irreduzibles}

\begin{karte}{Eisensteinkriterium}
    Seien \(R\) ein kommutativer nullteilerfreier Ring und \(P \subseteq R\) 
    ein Primideal. Weiter sei \( f = \sum_{i=0}^d r_i X^i \in R[X] \) 
    ein nichtkonstantes Polynom, dessen Leitkoeffizient 
    \(r_d\) nicht in \(P\) liegt, alle anderen Koeffizienten 
    aber schon. Schließlich sei \(r_0\) kein Produkt von zwei Elementen 
    aus \(P\).\\
    Dann ist \(f\) kein Produkt von zwei Faktoren in \(R[X]\), 
    die kleineren Grad haben.
\end{karte}

\begin{karte}{Inhalt eines Polynoms}
    Sei \(R\) ein Hauptidealring. Der \textit{Inhalt} \(\Inh(f)\) 
    eines Polynoms \(f\in R[X], f \neq 0\), ist definiert als der 
    Inhalt seiner Koeffizienten. \\
    Ist \(K\) der Quotientenkörper von \(R\) und \(f \in K[X]\) 
    ein Polynom \(\neq 0\), so gibt es ein \(0 \neq r \in R\) 
    mit \(rf \in R[X]\). Wir definieren den Inhalt von \(f\) dann als 
    \[ \Inh(f) := r^{-1} \Inh(rf). \]
\end{karte}

\begin{karte}{Lemma von Gauß}
    Seien \(R\) ein Hauptidealring mit Quotientenkörper \(K\) und 
    \(f,g\in K[X]\) von Null verschieden. Dann gilt 
    \[ \Inh(fg) = \Inh(f) \cdot \Inh(g). \]
\end{karte}

\begin{karte}{Irreduzibilitätskriterium}
    Sei \(R\) ein Hauptidealring mit Quotientenkörper \(K\) 
    und \(f\in R[X]\) ein nichtkonstantes Polynom, das in \(R[X]\) 
    kein Produkt von Faktoren kleineren Grades ist. \\
    Dann ist \(f\) in \(K[X]\) irreduzibel.
\end{karte}

\subsection{Klassische Probleme}

\begin{karte}{Kreisteilungspolynom}
    Sei 
    \[ \Phi_N(X) = \prod_{k\in (\Z/N\Z)^\times} (X - t^k). \]
    Die Nullstellen dieses Polynoms sind genau die 
    primitiven \(n\)-ten Einheitswurzeln in \(\C\). Es gilt 
    \[ \prod_{d|N} \Phi_d(X) = X^N - 1. \]
\end{karte}
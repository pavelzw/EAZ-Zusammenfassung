\section{Gruppen}

\subsection{Magmen}

\begin{karte}{Magma}
    Ein \textit{Magma} (oder \textit{Verknüpfungsgebilde}) ist eine 
    Menge mit einer (fixierten) Verknüpfung. Also ein Paar \((M,*)\), 
    wobei \(M\) eine Menge und 
    \[ \abb{*}{M\times M}{M} \]
    eine Abbildung ist.

    Ein Magma \((M,*)\) heißt \textit{assoziativ}, wenn für alle 
    \(l,m,n\in M\) die Regel 
    \[ (l*m)*n=l*(m*n) \] 
    gilt. Man nennt \((M,*)\) dann auch eine \textit{Halbgruppe}.

    Ein Element \(e\in M\) heißt ein (beidseitiges) \textit{Neturalelement} 
    des Magmas \((M,*)\), wenn für alle \(m\in M\) die Regel 
    \[ m*e = e*m = m \]
    gilt. Eine Halbgruppe mit Neturalelement nennt man auch ein \textit{Monoid}.

    Ein Magma \((M,*)\) heißt \textit{kommutativ}, wenn für alle \(m,n\in M\) die 
    Regel 
    \[ m*n=n*m \] 
    gilt.

    Für \(A,B\subseteq M\) schreiben wir \(A*B := \set{a*b \;|\;a\in A, b\in B}\).
\end{karte}

\begin{karte}{Untermagma}
    Eine Teilmenge \(U \subseteq M\) heißt \textit{Untermagma}, falls 
    \(U*U\subseteq U\) gilt. Dadurch wird \((U,*)\) ein Magma.
    
    Assoziativität und Kommutativität vererben sich. 
    Der Durchschnitt einer beliebigen Familie \((U_i)_{i\in I}\) 
    von Untermagmen ist wieder ein Untermagma von \(M\).

    Für \(X \subset M\) sei \(\langle X \rangle_{Magma}\) das \textit{Magmenerzeugnis} von \(X\) in \(M\). 
\end{karte}

\begin{karte}{Homomorphismus}
    Seien \((M,*)\) und \((N,\diamond)\) zwei Magmen. 
    Ein \textit{Homomorphismus} (auch \textit{verknüpfungserhaltende Abbildung})
    von \(M\) nach \(N\) ist eine Abbildung \(\abb{\Phi}{M}{N}\), 
    sodass für alle \(m_1, m_2 \in M\) die Gleichung 
    \[ \Phi(m_1*m_2) = \Phi(m_1) \diamond \Phi(m_2) \]
    stimmt.

    Das Bild \(\Phi(M)\) ist dann ein Untermagma von \(N\). 
    Das Urbild \(\Phi^{-1}(U)\) eines Untermagmas von \(N\) ist ein Untermagma von \(N\).

    Ist \(\Phi\) bijektiv, so nennt man ihn einen \textit{Isomorphismus}. 
    Dann ist auch \(\Phi^{-1}\) ein Homomorphismus. 
    
    Falls \(M=N\), haben wir \textit{Endomorphismen} und im bijektiven 
    Fall \textit{Automorphismen}. 

    Homomorphismen sind unter Komposition abgeschlossen.
\end{karte}

\subsection{Der Gruppenbegriff}

\begin{karte}{Gruppe}
    Es sei \((M,*)\) ein Magma. Dann heißt das Paar \((M,*)\) 
    eine \textit{Gruppe}, wenn es assoziativ ist, ein neutrales 
    Element \(e\) existiert und für jedes \(x\in M\) ein 
    \(y\in M\) existiert, sodass 
    \[ x*y=y*x=e \]
    gilt.

    Eine Gruppe ist \textit{kommutativ} oder \textit{abelsch}, 
    wenn sie als Magma kommutativ ist.
\end{karte}

\begin{karte}{Untergruppe}
    Sei \((G,*)\) eine Gruppe. Dann ist eine \textit{Untergruppe} von 
    \(G\) ein nichtleeres Untermagma \(U\), das unter Inversenbildung 
    abgeschlossen ist. \(U\subseteq G\) ist genau dann eine 
    Untergruppe, wenn \(U\) nicht leer ist und 
    \[ \forall x,y\in U: x y^{-1}\in U. \]

    Seien \((U_i)_{i\in I}\) Untergruppen von \(G\). Dann ist auch \(\bigcap_{i\in I} U_i\) 
    eine Untergruppe von \(G\).
\end{karte}

\begin{karte}{Gruppenerzeugnis, zyklische Gruppe}
    Für eine Teilmenge \(M\) der Gruppe \(G\) sei \(I\) die 
    Menge aller Untergruppen, die \(M\) enthalten. Dann ist auch 
    \[ \langle M \rangle := \bigcap_{i\in I} i \]
    eine Gruppe, sie heißt das (Gruppen-)Erzeugnis von \(M\) 
    oder die \textit{von \(M\) erzeugte Untergruppe von \(G\)}.
    Es gilt 
    \[ \langle M \rangle = \set{ x_1 \cdots x_k 
    \;|\; k\in \N_0, \forall i\leq k: x_i \in M \text{ oder } x_i^{-1} \in M }. \]

    Eine Gruppe heißt \textit{zyklisch}, wenn es ein Element \(a\in G\) 
    gibt, sodass \(G = \langle \set{a} \rangle =: \langle a \rangle\). 
\end{karte}

\begin{karte}{Ordnung}
    Die Kardinalität einer Gruppe nennt man auch ihre \textit{Ordnung}. 
    Die \textit{Ordnung eines Elementes \(g\in G\)} ist definiert als die Ordnung der von 
    \(g\) erzeugten Untergruppe.
    Wenn \(g\in G\) endliche Ordnung hat, dann ist diese gleich 
    der kleinsten natürlichen Zahl \(k\), für die \(g^k = e_G\) gilt.
\end{karte}

\begin{karte}{Satz von Lagrange}
    Sei \(G\) eine endliche Gruppe und \(H\) eine Untergruppe von 
    \(G\). Dann ist die Ordnung von \(H\) ein Teiler der Ordnung von \(G\).
\end{karte}

\begin{karte}{Index}
    Wir definieren auf \(G\) die Relation \(\sim\) durch 
    \[ g_1 \sim g_2 :\Leftrightarrow g_1 g_2^{-1} \in H. \]

    Wenn \(H\leq G\) zwei Gruppen sind, dann heißt die Anzahl 
    der Äquivalenzklassen auch der \textit{Index} von \(H\in G\): 
    \((G:H)\). 
    Es gilt für endliche Gruppen: 
    \[ \#G = \#H \cdot (G:H). \]
\end{karte}

\begin{karte}{Gruppenhomomorphismus}
    Seien \((G,*)\) und \((H,\bullet)\) zwei Gruppen. 
    Ein \textit{(Gruppen-)Homomorphismus} von \(G\) nach \(H\) 
    ist eine Abbildung \(\abb{f}{G}{H}\), für die gilt: 
    \begin{enumerate}
        \item \(\forall x,y\in G: f(x*y)= f(x)\bullet f(y)\).
        \item \(f(e_G) = e_H\).
        \item \(\forall x\in G: f(x^{-1}) = f(x)^{-1}\).
    \end{enumerate}

    Die Menge aller Homomorphismen von \(G\) nach \(H\) nennen 
    wir \(\Hom(G,H)\).
\end{karte}

\begin{karte}{Eigenschaften von Homomorphismen}
    Seien \(G\) und \(H\) Gruppen und \(\abb{f}{G}{H}\) ein 
    Magmenhomomorphismus. Dann gelten die folgenden Aussagen:
    \begin{enumerate}
        \item \(f\) ist ein Gruppenhomomorphismus. 
        \item \(f^{-1}(\set{e_H})\) ist eine Untergruppe von \(G\).
        \item \(f(G)\) ist eine Untergruppe von \(H\).
        \item \(f\) ist genau dann injektiv, wenn \(f^{-1}(\set{e_H}) = \set{e_G}\).
    \end{enumerate}
\end{karte}

\begin{karte}{Kern}
    Ist \(\abb{f}{G}{H}\) ein Homomorphismus zwischen zwei Gruppen, 
    so heißt die Untergruppe \(f^{-1}(\set{e_H}) \subseteq G\) der 
    \textit{Kern} von \(f\).\\
    \(f\in \Hom(G,H)\) ist genau dann injektiv, wenn \(\Kern(f) = \set{e_G}\).
\end{karte}

\begin{karte}{Konjugation}
    Sei \(G\) eine Gruppe. Für festes \(g\in G\) ist die Abbildung 
    \[ \abb{\kappa_g}{G}{G}, \kappa_g(x) := g x g^{-1} \]
    ein Automorphismus von \(G\). Sie heißt \textit{Konjugation} mit \(g\).
    
    Zwei Gruppenmitglieder \(x,y\in G\) heißen \textit{zueinander konjugiert}, 
    wenn es ein \(g\in G\) gibt mit \(y=g x g^{-1}\).

    Die Abbildung \(\abb{\kappa}{G}{\Aut(G)}, g\mapsto \kappa_g\), 
    ist ein Homomorphismus. Ihr Kern heißt das \textit{Zentrum} 
    \(Z(G)\) von \(G\), es gilt also 
    \[ Z(G) = \set{g\in G \;|\; \forall x\in G: gx=xg}. \]
    Das Bild \(\kappa(G) =: \mathrm{Inn}(G)\) wird die Untergruppe 
    der \textit{inneren Automorphismen} in \(\Aut(G)\) genannt.
\end{karte}

\begin{karte}{Normalteiler}
    Ein \textit{Normalteiler} in einer Gruppe \(G\) ist 
    eine Untergruppe \(N\), sodass für alle \(n\in N\) und 
    \(g\in G\) die Bedingung \(g n g^{-1} \in N\) erfüllt ist (\(U \lhd G\)). 
    Anders gesagt: \(N\) ist invariant unter allen inneren Automorphismen.\\
    Dann gilt sogar \(\forall g\in G: g N g^{-1} = N\).\\
    Jeder Kern ist ein Normalteiler.
\end{karte}

\subsection{Faktorgruppen}

\begin{karte}{Nebenklassen}
    \begin{enumerate}
        \item Sei \(G\) eine Gruppe und \(U\leq G\). Dann heißen 
        \(g,h\in G\) \textit{kongruent modulo \(U\)}, wenn 
        \[ g^{-1}h\in U. \]
        Das ist eine Äquivalenzrelation auf \(G\), die Äquivalenzklassen 
        \(gU = \set{gu \;|\; u\in U}\) heißen \textit{Linksnebenklassen} 
        nach \(U\). 
        Die Menge dieser Nebenklassen heißt der \textit{Faktorraum} \(G/U\).\\
        Die Abbildung \( \abb{\pi_U}{G}{G/U}, g\mapsto gU \) heißt 
        die \textit{kanonische Projektion}.
        \item Analog gibt es auch Rechtsnebenklassen \(Ug\), 
        die ebenfalls eine disjunkte Zerlegung von \(G\) liefern.
        \item \(Ug = gU\) für alle \(g\in G\) genau dann, wenn 
        \(U\) ein Normalteiler von \(G\) ist.
    \end{enumerate}
\end{karte}

\begin{karte}{Faktorgruppe}
    Es sei \(N\lhd G\) ein Normalteiler in \(G\). Dann wird auf \(G/N\) 
    durch 
    \[ (gN) \cdot (hN) := ghN \]
    eine Verknüpfung definiert. Somit bildet \(G/N\) eine Gruppe, 
    die \textit{Faktorgruppe von \(G\) modulo \(N\)}. \(\pi_N\) 
    ist ein hierbei ein Gruppenhomomorphismus. 
    Der Kern ist \(N\). Jeder 
\end{karte}

\begin{karte}{Homomorphiesatz}
    Seien \(G,H\) zwei Gruppen und \(N\lhd G\) ein Normalteiler. 
    \begin{enumerate}
        \item Die Abbildung 
        \[ \abb{L}{\Hom(G/N, H)}{\Hom(G,H)}, \Psi \mapsto \Psi \circ \pi_N \]
        ist injektiv und besitzt als Bild die Menge aller 
        \(\Phi \in \Hom(G,H)\) mit der Eigenschaft 
        \[ N\subseteq \Kern(\Phi). \]
        \item Ist \(\abb{\Phi}{G}{H}\) ein Homomorphismus mit \(\Kern(\Phi)=N\), 
        dann ist 
        \[ \tilde{\Phi}:G/N\ni gN \mapsto \Phi(g) \in \Bild(\Phi) \]
        ein Isomorphismus zwischen \(G/N\) und \(\Bild(\Phi)\).
    \end{enumerate}
\end{karte}

\begin{karte}{Erster Isomorphiesatz}
    Es seien \(G\) eine Gruppe, \(H\leq G\) eine Untergruppe und 
    \(N \lhd G\) ein Normalteiler. 
    Dann ist auch \(HN = \set{hn \;|\; h\in H, n\in N}\) eine Untergruppe 
    von \(G\) und es gibt einen Isomorphismus 
    \[ H/(N\cap H) \cong (HN)/N. \]
\end{karte}

\begin{karte}{Einfachheit}
    Eine Gruppe \(G\) heißt \textit{einfach}, wenn 
    sie nichttrivial ist und keine Normalteiler außer \(G\) und \(\set{e_G}\) besitzt. \\
    Eine nichttriviale Gruppe ist also genau dann einfach, wenn jeder nichtkonstante 
    Homomorphismus, der auf ihr definiert ist, injektiv ist.\\
    Eine abesche Gruppe ist genau dann einfach, wenn sie Primzahlordnung hat. 
\end{karte}

\begin{karte}{Direktes Produkt}
    Seien \(G\) und \(H\) zwei Gruppen. Dann ist auch die Menge 
    \(G \times H\) mit komponentenweiser Verknüpfung, also 
    \[ \forall (g,h), (g',h') \in G\times H: (g,h)\cdot (g',h') := (gg', hh') \] 
    eine Gruppe. Sie heißt das \textit{direkte Produkt} von \(G\) und \(H\).
\end{karte}

\begin{karte}{Freie Gruppe}
    Sei \(S\) eine Menge. 
    Eine \textit{freie Gruppe} über \(S\) ist eine Gruppe 
    \(F\) mit einer Abbildung \(\abb{f}{S}{F}\), sodass für jede 
    Gruppe \(G\) und jede Abbildung \(\abb{\varphi}{S}{G}\) genau ein 
    Gruppenhomomorphismus \(\abb{\Phi}{F}{G}\) existiert, für den 
    \[ \forall s\in S: \varphi(s) = \Phi(f(s)) \]
    gilt.

    Analog das \textit{freie Monoid} \(M\) über \(S\)
    als ein Monoid \(M\) mit einer Abbildung \(\abb{f}{S}{M}\), sodass 
    für jede Abbildung von \(S\) in ein Monoid \(N\) genau ein Monoidhomomorphismus 
    von \(M\) nach \(N\) existiert, sodass das folgende Diagramm kommutiert:
    \begin{center}
    \begin{tikzcd}
        & M \arrow[dashed]{dd}{\Phi} \\
        S \arrow{ru}{f} \arrow{rd}{\varphi} \\
        & N
    \end{tikzcd}
    \end{center}

    Es gilt Eindeutigkeit und Existenz.
\end{karte}

\begin{karte}{Erzeuger von Relationen}
    Jede Gruppe \(G\) lässt sich als Faktorgruppe einer freien 
    Gruppe schreiben, notfalls nehme man die freie Gruppe \(F\) über 
    \(G\) und setze die Identität auf \(G\) zu einem Gruppenhomomorphismus 
    von \(F\) nach \(G\) fort.
\end{karte}

\subsection{Gruppenoperationen}

\begin{karte}{Gruppenoperation}
    Sei \((G,*)\) eine Gruppe und \(M\) eine Menge. Dann 
    ist eine \textit{Operation von \(G\) auf \(M\)} definiert als eine Abbildung 
    \[ \abb{\bullet}{G\times M}{M}, \] 
    sodass die folgenden Bedingungen erfüllt sind. 
    \begin{enumerate}
        \item \(\forall m\in M: e_G \bullet m = m\),
        \item \(\forall m\in M, g_1, g_2\in G: g_1 \bullet (g_2 \bullet m) 
        = (g_1 * g_2) \bullet m\).
    \end{enumerate}
    Eine Menge \(M\) mit einer festen Operation einer Gruppe \(G\) 
    heißt \(G\)-Menge. 
\end{karte}

\begin{karte}{Operationen und symmetrische Gruppe}
    Es seien \(G\) eine Gruppe und \(M\) eine Menge. 
    \begin{enumerate}
        \item Für jeden Homomorphismus \(\abb{\Phi}{G}{\Sym(M)}\) 
        wird durch 
        \[ g\bullet m := \Phi(g)(m) \]
        eine Operation von \(G\) auf \(M\) festgelegt. 
        \item Für jede Operation \(\bullet \) von 
        \(G\) auf \(M\) gibt es einen Homomorphismus \(\Phi\), 
        sodass \(\bullet\) wie in 1. konstruiert werden kann.
    \end{enumerate}
\end{karte}

\begin{karte}{Äquivalenzrelation Operation}
    Sei \(G\) eine Gruppe, die auf der Menge \(M\) operiert. Dann wird auf 
    \(M\) durch die Vorschrift 
    \[ m_1 \sim m_2 :\Leftrightarrow \exists g\in G: m_1 = g\bullet m_2 \]
    eine Äquivalenzrelation definiert.
\end{karte}

\begin{karte}{Bahnen, Transitivität, Stabilisator, Fixpunkt}
    Die Äquivalenzklassen der Äquivalenzrelation werden 
    \textit{Bahnen} oder \textit{Orbiten} genannt. Die \textit{Bahn} 
    von \(m\) wird als 
    \[ G\bullet m = \set{g\bullet m \;|\; g\in G} \]
    notiert.

    Die Operation heißt \textit{transitiv}, wenn es genau eine Bahn gibt, 
    also wenn ein \(m_0\) existiert, sodass für jedes \(m\in M\) ein 
    \(g\in G\) existiert mit der Eigenschaft 
    \[ m = g\bullet m_0. \]

    Der \textit{Stabilisator eines Elements \(m\in M\)} unter 
    einer gegebenen Operation der Gruppe ist definiert als 
    \[ \Stab_G(m) := \set{g\in G \;|\; g\bullet m = m}. \]

    Ein \textit{Fixpunkt von \(G\)} auf \(M\) ist ein Element, 
    dessen Stabilisator ganz \(G\) ist. Die Menge aller Fixpunkte 
    wird mit \(M^G\) notiert: 
    \[ M^G := \set{m\in M \;|\; \forall g\in G: g\bullet m=m}. \]
\end{karte}

\begin{karte}{\(G\)-äquivariante Abbildung}
    Wenn die Operation transitiv ist und \(m\in M\) irgendein 
    Element sowie \(H\leq G\) dessen Stabilisator, dann ist die Abbildung 
    \[ \abb{f}{G}{M}, f(g) := g\bullet m, \]
    surjektiv und auf den Linksnebenklassen von \(H\) konstant. 
    Sie legt also eine surjektive Abbildung 
    \(\abb{\tilde{f}}{G/H}{M}\) fest, für die gilt: 
    \[ \forall g\in G,x\in G/H: \tilde{f}(gx)=g\bullet \tilde{f}(x). \]
    Diese Abbildung ist dazu noch injektiv, also sind \(G/H\) und 
    \(G\bullet m\) gleich mächtig.

    Wenn \(M,N\) zwei Mengen mit \(G\)-Operationen sind, so heißt eine Abbildung 
    \( \abb{f}{M}{N} \) mit \(f(gm) = gf(m)\) eine \textit{\(G\)-äquivariante Abbildung}. Die Abbildung 
    \(\tilde{f}\) ist ein Beispiel hierfür.
\end{karte}

\begin{karte}{Bahnbilanzformel}
    
\end{karte}

\begin{karte}{}

\end{karte}

\begin{karte}{}

\end{karte}
